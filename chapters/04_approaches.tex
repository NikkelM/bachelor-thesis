\begin{jointwork}\label{ch:Approaches}
	In this section we will take a look at the different approaches we took to monitoring agents in our framework, explaining the reasons why we chose to implement specific features and how they help us in determining an agents strengths and weaknesses.
\end{jointwork}

\section{When to monitor what}\label{sec:WhenToMonitorWhat}

Our \emph{workflow} (which will be explained in more detail in \nameref{ch:OurWorkflow}) can generally be split into two parts when it comes to monitoring and evaluating agents: \emph{during} and \emph{after} training. When talking about the \emph{workflow} we refer to the process of configuring and starting a training session, where a Reinforcement-Learning agent is being trained on a specific marketplace against competitors. The \emph{workflow} also includes the subsequent collection of data used to evaluate an agent's performance. We are also introducing the term of the \emph{complete agent} in this section, which will be used to refer to both Reinforcement-Learning agents that have completed a training run, as well as Rule-Based agents which do not need training.

As mentioned above, we split our monitoring tools into the following two categories:

\begin{enumerate}
	\item \textbf{During training}: Having data available as soon as possible without having to wait for a long training session to end is crucial to an efficient workflow. Our framework enables us to collect and visualize data while a training session is still running. This allows users to always be in the know about the currently running experiments. In some cases, when an agent's performance is severely lacking, users may want to stop a training session before it has finished, which is enabled through these monitoring tools. We also include the \emph{Live-monitoring} tool in this category, which runs directly after a training session has finished, see \nameref{subsec:LiveMonitoring}.

	\item \textbf{On complete agents}: After a training session has finished we have a complete and final set of data available for an agent, which enables us to perform more thorough and reliable tests. These can include simulating runs of a marketplace to gather data on the agent's performance in different scenarios and against different competitors, or running a static analysis of the agent's policy in different market states. The tools available for trained agents are in the same way also usable on Rule-Based agents.
\end{enumerate}

In \todo{If we also conduct experiments in this chapter, mention it here}the following sections, we will take a look at all of the tools our framework provides for monitoring agents, distinguishing between the two general types of monitoring mentioned above. The goal of these sections is to give a short overview of each tool, how and why they were implemented and what value they offer to the framework as a whole. We will also discuss some features that are not yet available, explaining how they could benefit the entire workflow and enrich the overall experience.

\section{Monitoring a training session}

When talking about monitoring agents during a training session, we are always referring to Reinforcement-Learning agents, as Rule-Based agents always perform the same and cannot be trained. But even though they cannot be trained, our monitoring tools listed in the next section, \nameref{sec:CompleteAgents}, can still be applied to Rule-Based agents as well, as users may want to compare different Rule-Based strategies against each other, or measure the strategy's performance on a market before training a Reinforcement-Learning agent.

Monitoring agents while they are still being trained enables us to be more closely connected to the training process. Ultimately, the goal of such monitoring tools is to be able to predict the estimated `quality' of the final trained agent as reliably as possible while the training is still going.

\subsection*{TensorBoard}\label{subsec:TensorBoard}

The \emph{TensorBoard} is an external tool from the from the Reinforcement-Learning library \emph{TensorFlow}~\cite{TensorFlow}. With just a few lines of code a training session can be connected to a TensorBoard instance. We are then able to pass any number of parameters and metrics we deem interesting or useful to the TensorBoard, which then offers visualizations for each of them, updating live as the training progresses. To access these web-based visualizations, a local server needs to be started. The diagrams created using the TensorBoard are an immensely useful tool for quickly and easily recording data and offering a first rough comparison of competitors in the market. Aside from simple visualizations, TensorBoard also offers many plugins, and even enables users to write their own~\cite{TensorBoardPlugins}. Plugins such as the \emph{What-If Tool} (\cite{WhatIfTool},~\cite{WhatIfToolWeb}), which allows users to feed trained models with hypothetical situations to study their behaviour, can have a great influence on the way users interact with the TensorBoard and their machine learning models.\todo{List some/all of the data provided through tensorboard. Maybe some here and the rest in appendix table}

\subsection*{Live-monitoring}\label{subsec:LiveMonitoring}

Unlike the TensorBoard, the monitoring tools summarised under the term \emph{Live-monitoring} were completely and from the ground up built by our team. For most of the visualizations, the \emph{matplotlib}~\cite{Matplotlib} library was used. The Live-monitoring tool combines two use-cases: First, it creates visualizations for all data recorded during the training, similar to those provided through the TensorBoard. This needs to be done to be able to access the visualizations even after the training has concluded, as the TensorBoard relies abstract data files for its visualizations. By taking the data we (still) have at the end of the training and using our own visualization tool, we create two types of diagrams: Scatterplots, which contain all samples for a certain parameter (see for example \Cref{fig:SACDuopolyMixedGraphs}) and lineplots, which show smoothed data, such as it would be available in the TensorBoard (see for example \Cref{fig:SACDuopolyProfitsMean}). Secondly, the tool simulates a market scenario identical to the one used during training an additional time. To understand why we do this, we need some additional information: During a training session, `intermediate' models, as we will call them, are being saved in regular intervals. These models contain the current policy of the agent and can be used the same as any other model of complete agents, the only difference being the quality of the agent, which can change over the course of a training run, both for the better and worse. These intermediate models can then be used by a range of monitoring tools available to us. Since the models only contain the current policy of an agent but not the history of states and actions preceding the model, we need to run separate simulations on these models to be able to analyze and evaluate them. For this, we utilize our \emph{Agent-monitoring} toolset (explained in detail in \nameref{subsec:AgentMonitoring}). In \nameref{subsec:LiveMonitoringResults} we discuss the results of a training session using the Live- and Agent-monitoring tools.

\section{Monitoring complete agents}\label{sec:CompleteAgents}

For monitoring trained Reinforcement-Learning and Rule-Based agents, we offer three major tools: The \nameref{subsec:AgentMonitoring} tool allows users to simulate a large number of episodes to visualize bigger trends, the \nameref{subsec:Exampleprinter} simulates a single episode, offering detailed insights into market states using an overview diagram, and the \nameref{subsec:Policyanalyzer} is a static tool which can be used to analyze a vendor's reaction to different market states and competitor actions.

\subsection*{Agent-monitoring}\label{subsec:AgentMonitoring}

Using the \emph{Agent-monitoring} toolset, users can configure a custom market simulation, using the following parameters:

\begin{enumerate}
	\item Episodes: This parameter decides how many independent simulations are run in sequence. At the start of each episode, the market state will be reset and randomized. Within an episode, the vendors run through a configurable amount of timesteps, during each of which they set prices (depending on the chosen economy type, this can range from only one price for new items to three prices, including a re-buy price for used items) and a set number of customers interact with them.
	\item Plot interval: A number of diagram types enable the user to view averaged or aggregated data over a period of time. The plot interval parameter decides the size of these intervals. Smaller intervals mean more accurate but also more convoluted data points. Computational time also increases with a smaller interval size.\todo{These diagrams can be seen/are explained in...}
	\item Marketplace: Using this parameter, the user can set the marketplace on which the monitoring session will be run. See \nameref{sec:MarketScenarios} for an explanation of the different available marketplaces.
	\item Separate markets: This parameter is a boolean flag that determines the way in which the monitoring session will handle the agents given by the user. If the flag is enabled, each agent will be initialized on a separate instance of the chosen marketplace, meaning that the agents will be monitored independently from each other. This functionality takes a lot longer than if the flag were disabled, as the whole marketplace is simulated once for each agent. While it may seem like the same results could be reached by simply starting multiple monitoring sessions with a different agent each, this is not the case. Using this flag instead, it is ensured that all agents get the exact same market states for each episode. By running multiple marketplaces in parallel using the \emph{separate markets} flag, we can match the simulations as closely as possible. The most prominent use-case where this flag is enabled is during the Live-monitoring after a training session, where all intermediate models are being monitored on separate markets.\todo{Judith: `For whole section: Maybe highlight some keywords to make it more readable?'}

	If the flag is disabled, the monitoring tool will initialize only one marketplace and set the passed agents to directly compete against each other on this marketplace. This functionality is most useful when monitoring only a single agent, trying to determine its specific strengths and weaknesses against certain opponents, as it will complete a lot faster than if the flag were enabled.
	\item Agents:\todo{Judith: `Absatz ist etwas unstrukturiert'} Depending on the chosen marketplace, only a select number of agents can be chosen to be monitored, as each agent is built to interact with a specific type of marketplace. First off, all agents belong to one of the two major categories: \emph{Reinforcement-Learning agent} or \emph{Rule-Based agent} (for a more detailed overview see \nameref{sec:ExplainVendors}). Reinforcement-Learning agents can only be monitored on the marketplace type and market environment they were trained on, as these define the number of inputs and outputs the agent expects. Rule-Based agents can only be used on the marketplace type they were built for, as each of them makes assumptions about the number of prices they will need to set, but the market environment may be freely chosen. This leads to not all marketplace types having the same amount of Rule-Based vendors available. Following their importance for our simulation framework, the Linear Economy has the least and most often weakest vendors available, while the more refined competitors are most of the times only available as a version compatible with the Circular Economy with rebuy prices.
\end{enumerate}

During each episode and for each vendor, all market events are being recorded. At the end of a monitoring session, the collected data is evaluated in different visual formats. First of all, all data that would be available to see in the \emph{TensorBoard} during a training session is visualized using density plots. These plots can be used to compare the vendors in a more granular way, if for example the effect of a parameter on the customer's sell-back behaviour of used items should be tested. Another visualization that is created is a histogram containing the cumulative profits per episode for each agent, plotted against each other. This allows for a quick overview to see which agent had an overall better performance. One additional type of diagram is only created if the Agent-monitoring is run through the Live-monitoring tool: Violinplots. These plots, which are created for all datapoints available through TensorBoard,\todo{List} depict distributions using density curves, accentuating the minimum, maximum and median values. Violinplots are used by us to compare different training stages of an agent, as small policy changes can have great impact on these values.\todo{Reference to a violinplot in chapter 6}

Aside from monitoring after a training session, a common usecase of the \emph{Agent-monitoring} tool is to test trained agents against competitors other than the ones it was trained against. This is done to test the agent's capacity to adapt to different circumstances, an important factor in deciding an agent's quality, as its competitors in the real market will differ from any it has encountered in training, due to the sheer vastness of options when it comes to dynamic pricing models available nowadays, see \nameref{sec:ExplainVendors}.

\subsection*{Exampleprinter}\label{subsec:Exampleprinter}

The \emph{Exampleprinter} is a tool meant for quickly evaluating a market scenario in-depth. When run, each action taken by the monitored agents is being recorded, in addition to market states and events such as the number of customers arriving and the amount of products thrown away. At the end of this quick simulation an animated overview diagram is created\todo{Example svg somewhere}, which shows all actions and their consequences for each step in the simulation. Due to the large amount of data that is being collected and the overhead that would come with it, we chose to disconnect this functionality from large-scale tools such as the Agent-monitoring. While the Agent-monitoring tool could be seen as a tool that imitates Macro-economic behaviour, simulating hundreds of days through hundreds of episodes, the Exampleprinter instead simulates only one day, recording and visualizing all data collected during that time.

\subsection*{Policyanalyzer}\label{subsec:Policyanalyzer}

The last tool we want to introduce is the \emph{Policyanalyzer}. The \emph{Policyanalyzer} is our only tool which does not simulate a market in any way. Instead, the tool can be used to monitor an agent's reaction to different market states. The user can decide on up to two different features to give as an input, such as a competitor's new and used prices, and the Policyanalyzer will feed all possible input combinations to the agent and record its reactions. When initializing the \emph{Policyanalyzer}, the user defines a number of parameters: The agent whose policy should be analyzed, as well as the marketplace and the competitors that should be used, just as is done for all of the other tools. Additionally, the user defines a \textbf{template market state}, a market state containing all values that are passed to the analyzed agent, such as the number of items currently in circulation and the prices of competitors. Lastly, a list of \textbf{analyzed features} needs to be provided, which defines one or two features of the template market state that should be varied. When the \emph{Policyanalyzer} is run, these features are inserted into the template market state, overwriting the initial values and creating a new combination. This new market state is then passed on to the \texttt{policy}-method of the analyzed agent (for an example policy, see \Cref{lst:PolicyRuleBasedStorageMinimizer}), and its reactions are recorded and visualized, see \todo{Diagram}.

The \emph{Policyanalyzer} is the monitoring tool which operates on the smallest scale out of all the tools we built for our framework. It allows users to define any market state they want and to then accurately monitor a vendor's reactions to changes to this specific state. While the tool can just as well be utilized to test new Rule-Based strategies, it is very much meant to be used as a way to understand Reinforcement-Learning agents better, as their policies are not immediately visible to the end-user and must therefore be discovered through tools such as the one's we built. 

\section{Improving our tools}

While all of the tools we currently have available are able to do what is expected of them and they each have their own strengths, there are still a number of ways that the different tools could be enhanced. In the following sections, we introduce a number of ways in which the different monitoring tools of our simulation framework could be enhanced.

Before going into detail about the different tools, there is one enhancement which concerns all of our tools. As some tools are older than others and not only the tools itself but also the way we simulate our market has changed over time during development, not all tools record the same data. For example, the Exampleprinter records all prices set by vendors, but does not show a breakdown of profits between the different retail channels. On the other hand, the Agent-monitoring does offer such a breakdown, but does not record the vendors' actions and thereby prices. This can lead to confusion, as users may expect a certain type of diagram to be created when using one tool or the other. So an important, though perhaps time-intensive, enhancement would be to \textbf{unify the data that is being recorded across all monitoring tools}.

\subsection*{Live-monitoring}\label{subsec:FutureLiveMonitoring}

Currently, the Live-monitoring tool works by first taking and visualizing all the data collected during a training run, and then running the Agent-monitoring tool on the saved intermediate models. The biggest downside to this is that users need to wait until the training session has finished before the Agent-monitoring is run, meaning that a lot of potential information is lost. Imagine a scenario where a training session was initialized to run for 10000\todo{Should I put delimiters in large numbers? (10,000)} episodes, saving intermediate models each 1000 episodes. In the end, when the Live-monitoring tool is run, the user may find that the best model was the one saved after 3000 episodes, meaning that a lot of time was wasted training an additional 7000 episodes. The possibility of this happening may at first seem counterintutitive, but is a common phenomenom when training Reinforcement-Learning agents, known as \emph{Catastrophic Forgetting} (see also \cite{CatastrophicForgetting}). By improving the Live-monitoring tool with an option to \textbf{run the Agent-monitoring whenever an intermediate model is saved}, the user would be able to recognize such trends much faster and terminate the experiment at the right time. This feature could be further enhanced with a smart built-in option that \textbf{terminates the experiment for the user if a downward trend in performance is detected}. Specific thresholds for these terminations should also optionally be \textbf{set by the user}. For this, the data created during the Agent-monitoring tool would need to be saved in a machine-readable form - graphs and diagrams are not useful here.

\subsection*{Agent-monitoring}\label{subsec:FutureAgentMonitoring}

This brings us to improvements that could be made to the Agent-monitoring tool. At the moment, a lot of data is recorded when simulating the marketplace, but only graphs and diagrams are created as a result of the simulation. By \textbf{giving users the option to have data saved as (for example) \texttt{.csv} files}, users would be enabled to use the results of the simulation in other ways more easily, even for monitoring and evaluation tools completely disconnected from our own framework and the tools we provide. Reproducability is also a major concern when it comes to evaluating simulation results, as was already mentioned in \nameref{ch:RelatedWork} and is discussed in many papers such as~\cite{DRLThatMatters} and~\cite{ReproducabilityRL}. In our simulation framework, with the start of a new episode the market state is always shuffled randomly, to allow Reinforcement-Learning algorithms to properly explore the environment. This is however creating the problem of creating simulations which are currently impossible to reproduce, which could be solved by \textbf{introducing a seed-based system for shuffling market states and sharing this seed with the user}. When running a different simulation with the same seed, assuming that the marketplace type and evironment stay the same, users can recreate the same random market states that are set at the start of an episode, allowing for even better and in-depth comparisons of different vendors, past a single run of the Agent-monitoring tool.

\subsection*{Exampleprinter}\label{subsec:FutureExampleprinter}

The Exampleprinter is a great tool for quickly monitoring and evaluating a certain market setup. However, only for one specific combination of marketplace type and market environment, a Duopoly scenario of a Circular economy with rebuy prices, the animated overview diagram is created. \textbf{Building templates and adding diagram support for more scenarios} should therefore be a priority when enhancing the Exampleprinter. Additionally, the \textbf{market-seed feature} introduced in \nameref{subsec:FutureAgentMonitoring} should also be implemented for the Exampleprinter, to allow users to run a simulation multiple times to monitor possible discrepancies in agent behaviour and find outliers in the data. This could be further enhanced by adding a configuration option that allows for \textbf{more than one episode to be simulated at a time}.

As a combined enhancement for both the Agent-monitoring and the Exampleprinter, the \textbf{Exampleprinter could be integrated into the Agent-monitoring}, while still being its own tool, the same as the Agent-monitoring is integrated into the Live-monitoring.

\subsection*{Policyanalyzer}\label{subsec:FuturePolicyAnalyzer}\todo{Some wording in the beginning of this must be changed if chapters 4 and 5 are swapped}

The biggest and most important improvement to the Policyanalyzer does not concern its concrete features, but the way users interact with it. Currently, all of the other monitoring tools are either integrated into some part of the workflow (e.g. the Live-monitoring), or can easily be started using simple commands, see \nameref{ch:OurWorkflow}. This is however not the case for the Policyanalyzer, which must be started by going into the code itself, which is less than ideal from a user-perspective. So, \textbf{integrating the Policyanalyzer into the workflow}, both by \textbf{creating a user-facing interface} for it and by \textbf{adding configuration options to start it after a training session} are features that should be a priority when continuing work on the framework. Users are also able to use the Policyanalyzer to analyze a large number and combination of features, so \textbf{curating a list of useful feature combinations to be analyzed} would aid many users when using this tool.