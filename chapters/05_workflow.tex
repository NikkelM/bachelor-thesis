\label{chapter:OurWorkflow}

\begin{jointwork}
	The main goal of the bachelor's project is to provide a simple-to-use but powerful interface for training Reinforcement-Learning algorithms on highly configurable markets for users in both a research and a business context. To achieve this, multiple components \todo{Better word for components}had to be developed and connected to create the workflow we now provide. This section will go over the most important parts of the workflow, focusing on the way each of them supports the monitoring capabilities of the framework.
\end{jointwork}

\section{Configuring the run}

Unarguably\todo{Is this "zu wertend"?}, the most important part of the whole workflow is its configuration. Without it, each simulation and training session would produce similar, if not the same results. By tweaking different parameters of a run, market dynamics can be changed and agent performance be influenced. The goal of our monitoring tools is to enable users to assess the extent to which each changed parameter may have influenced certain characteristics of the training and/or monitoring session, and to enable them to make informed decisions for subsequent experiments.

\subsection{The webserver/Docker-API}

\todo{Example screenshot of the webserver. Configuration and running page?}


\section{During training}
\subsection{Choosing what to show the user}
\section{Saving models at certain stages during training}
\subsection{Live-monitoring tools (see Chapter 4)}

\section{After training/Complete agents}
\subsection{Why do we even monitor these agents?}
\subsection{Which datapoints prove to be/are most effective?}