\begin{jointwork}\label{ch:RelatedWork}
	This section will outline approaches to modelling and simulating recommerce marketplaces, as well as give an introduction into the concepts of Reinforcement-Learning, the technology the framework is built for. Additionally, novel and unconventional approaches to monitoring and evaluation of these Reinforcement-Learning agents will be discussed.\todo{Do you `discuss' in the context of related work?}
\end{jointwork}

\section*{Market simulations}

\section*{Reinforcement-Learning}

\section*{Visualization - State-of-the-art}

The process of training Reinforcement-Learning agents for any kind of task always comes with the requirement for visualizing the data collected during training. This allows for an analysis of the algorithm's performance and gives insights into its strengths and weaknesses.

For the past years, going back as far as 2018, one of the most used frameworks overall and the most used for machine learning was TensorFlow~\cite{StackOverflowSurvey}. Aside from its API for model building, TensorFlow also provides a visualization toolkit called TensorBoard, which can be used independent of other TensorFlow tools. TensorBoard provides an API for tracking and visualizing important metrics such as loss and accuracy, and allows developers to easily integrate their own metrics as well. During an experiment, data can be visualized live during the training process, allowing developers to quickly gain insights into the performance of the algorithms. In our recommerce market simulation framework, we use the TensorBoard in conjunction with our own tools for data visualization, see \nameref{ch:Approaches}\todo{Do I refer to our own project in the related work chapter?}\todo{Do I need more references for TensorBoard, or some pictures?}.

\section*{Visualization - Novel approaches}