\begin{table}[ht]
	\begin{tabular}{p{0.35\textwidth} p{0.65\textwidth}}
		\toprule
		Consumer Characteristic                 & Description                                                                                                            \\\midrule
		Perfectionistic, High-Quality Conscious & Consumer searches carefully and \newline systematically for the very best quality in products                          \\
		Brand Conscious, `Price = Quality'      & Consumer is oriented towards buying the more expensive, well-known brands                                              \\
		Novelty and Fashion Conscious           & Consumers who like new and innovative products and gain excitement from seeking out new things                         \\
		Recreational and Shopping Conscious     & Consumer finds shopping a pleasant activity and enjoys shopping just for the fun of it                                 \\
		Price Conscious/ Value for the Money    & Consumer with a particularly high consciousness of sale prices and lower prices in general                             \\
		Impulsive/ Careless                     & Consumer who buys on the spur of the moment and appears unconcerned about how much he/she spends                       \\
		Confused by Overchoice                  & Consumer perceiving too many brands and stores from which to choose and experiences information overload in the market \\
		Habitual/ Brand Loyal                   & Consumer who repetitively chooses the same favorite brands and stores                                                  \\\bottomrule
	\end{tabular}
	\caption{Consumer Shopping Styles, taken from~\cite{ShoppingStyles}, including information from~\cite{ShoppingStyles2}}\label{tab:shoppingStyles}
\end{table}

\newpage

\begin{table}[ht]
	\begin{tabular}{p{1\textwidth}}
		\toprule
		I - Economic dimensions                                                              \\\midrule
		1. ECO1 - Buying cheaper, spending less (anxiety expressed in regard to expenditure) \\
		2. ECO2 - Paying fair prices                                                         \\
		3. ECO3 - Allocative role of price (what is obtained for a particular budget)        \\
		4. ECO4 - Bargain hunting                                                            \\
		\toprule
		II - Dimensions relating to the nature of the offering                               \\\midrule
		5. OFF1 - Originality                                                                \\
		6. OFF2 - Nostalgia                                                                  \\
		7. OFF3 - Congruence                                                                 \\
		8. OFF4 - Self-expression                                                            \\
		\toprule
		III - Dimensions relating to the recreational aspects of second-hand channels        \\\midrule
		9. CIR1 - Social contact                                                             \\
		10. CIR2 - Stimulation                                                               \\
		11. CIR3 - Treasure hunting                                                          \\
		\toprule
		IV - Power dimensions                                                                \\\midrule
		12. PUIS1 - Smart shopping                                                           \\
		13. PUIS2 - Power over the seller                                                    \\
		\toprule
		V - 14. ETH - Ethical and ecological dimension                                       \\
		\toprule
		VI - 15 ANT-OST - Anti-ostentation dimension                                         \\
		\bottomrule
	\end{tabular}
	\caption{15 areas of motivation toward second-hand shopping, taken from~\cite{SecondHandMotives} (descriptions omitted)}\label{tab:SecondHandMotives}
\end{table}

\newpage

\lstset{language=Python}
\lstset{frame=lines}
\lstset{caption={Policy implementation of the \emph{RuleBasedCEAgent}}}
\lstset{label={lst:PolicyRuleBasedCE}}
\lstset{basicstyle=\footnotesize}
\begin{lstlisting}
def policy(self, market_state) -> tuple:
  products_in_storage = market_state[1]
  price_refurbished = 0
  price_new = self.config_market.production_price
  rebuy_price = 0
  if products_in_storage < self.config_market.max_storage/15:
    # fill up the storage immediately
    price_refurbished = int(self.config_market.max_price * 6/10)
    price_new += int(self.config_market.max_price*6/10)
    rebuy_price = price_refurbished-1

  elif products_in_storage < self.config_market.max_storage/10:
    # fill up the storage
    price_refurbished = int(self.config_market.max_price * 5/10)
    price_new += int(self.config_market.max_price * 5/10)
    rebuy_price = price_refurbished - 2

  elif products_in_storage < self.config_market.max_storage/8:
    # storage content is ok
    price_refurbished = int(self.config_market.max_price * 4/10)
    price_new += int(self.config_market.max_price * 4/10)
    rebuy_price = price_refurbished // 2

  else:
    # storage too full, get rid of some refurbished products
    price_refurbished = int(self.config_market.max_price * 2/10)
    price_new += int(self.config_market.max_price * 7/10)
    rebuy_price = 0

  price_new = min(9, price_new)
  return (price_refurbished, price_new, rebuy_price)
\end{lstlisting}