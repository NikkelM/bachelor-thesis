\begin{jointwork}
	In this section we will outline the requirements and challenges of transferring the vast amount of features and parameters that influence vendor and customer decisions in real recommerce markets to our simulated market environment. We will take a brief look at different market scenarios as well as how our customers work, with a focus on the unique feauture of recommerce markets; the buyback of used items by the vendors. The focus of this chapter will however lie on the vendors, the rule based as well as the ones trained using Reinforcement Learning, comparing their features and how they fare against each other. These comparisons will be done using our various monitoring tools, which will be explained in the following chapter, \nameref{ch:Approaches}. While results of these experiments will be shortly touched upon, we will go into further detail in the closing chapter, \nameref{ch:InterpretResults}.
\end{jointwork}

\section{Market scenarios} \label{sec:MarketScenarios}

In our framework, we implemented a number of "market blueprints" both for classic Linear Economy markets, as well as for Circular Economy markets both with and without rebuy-prices enabled. All market types offer various configurations for the number of competitors: Monopoly, Duopoly and Oligopoly scenarios can be configured, with the Duopoly and Oligopoly scenarios offering rulebased competitors specifically built for the respective market scenario.

\section{Customers}

Customers are at the center of every type of marketplace, which makes them an integral part of our market simulation. However, since each customer in the real market is an individual with different reasonings and makes purchase decisions based upon personal preferences, modelling a realistic depiction of real-world customers proves to be very difficult and time-consuming. For this reason we decided to keep our initial implementation of the customers as simple as possible, taking into account future extension and scalability concerns. As we are focussing on dynamic pricing and therefore allow our vendors to only change the prices of their products, we decided on building customers that value price over any other features a product may have. Additionally, in Circular Economy markets, customers have preferences on whether they would like to buy new or refurbished items.

Once customers have decided to buy a product from any of the available vendors, they turn into an \emph{owner}. In the next step of the simulation, they are offered the option of selling their now used product back to one of the vendors. If they decide to do so, the vendor pays them the advertised \emph{re-buy price} and adds the used product to their inventory, from where it can then be sold as a refurbished procuct in the next step.

\section{Vendors} \label{sec:ExplainVendors}

Vendors are the main focus of our market simulation.

While our framework will not be able to reproduce all types of pricing models used in the real market, we strive to model as many different models as possible (and feasable in the scope of the project). The article "Dynamic pricing models for electronic business" \cite{dynamicPricingModels} has grouped a few of the most common types of dynamic pricing models into the following five categories, which we will explore in the context of our framework:

\begin{enumerate}
	\item \emph{Inventory-based models}: These are pricing models which are based on inventory levels and similar parameters, such as the number of items in circulation. In our framework, almost all rule-based agents consider their inventory levels when deciding which prices to set. The only exception to this rule are the simplest of our agents, the \emph{FixedPriceAgents}, which will always set the same prices, no matter the current market state and competitor actions. We chose to implement this rather incompetent type of agent, as there are still many vendors in the real market which do not make use of dynamic pricing models, but use static pricing methods like these \emph{FixedPriceAgents} instead. Not including these in our framework would therefore remove a fraction of realism.

	      Inventory-based models are comparatively easy to implement, as they only depend on data immediately available to the vendor. This has the advantage that rule based agents based on this technique are relatively simple to create and modify. Examples of \emph{Inventory-based agents} in our framework can be found in the \emph{RuleBasedCEAgent}, one of \emph{RuleBasedCEAgent} the first rule based agents we created. This agent does not take pricing decisions of its competitors in the market into account, but simply acts according to its own storage costs. While its performance is not necessarily bad, it is still one of the weakest competitors currently available in the framework. \todo{Compare this agent with other agents, RL, FixedPrice, better rule-based!}

	\item \emph{Data-driven models}: \emph{Data-driven} models take dynamic pricing decisionmaking one level further. They utilize their knowledge of the market, such as customer preferences, past sales data or competitor prices, to derive optimal pricing decisions. Aside from the aforementioned \emph{FixedPriceAgents} and the basic \emph{RuleBasedCEAgent}, all of our other rule based agents fall into this category. One of the most prominent examples of a \emph{Data-driven model} is the \emph{RuleBasedCERebuyAgentStorageMinimizer}. This agent bases its prices on two major factors. First, it reacts to its competitors prices by matching its initial prices to the median price of its competitors, and then adjusts them according to current inventory. Notably, all of our \emph{Data-driven models} are also \emph{Inventory-based} to a certain extent, as handling storage plays a big part in a Circular Economy market setting, where used products need to be bought back from customers and subsequently undergo refurbishment while in inventory of the company. \emph{Data-driven models} have proven to be the most competent rule based agents in our recommerce market scenario, in particular the above described \emph{RuleBasedCERebuyAgentStorageMinimizer}, which only consists of ten lines of code but is still able to outperform seemingly more complex \emph{Inventory-based models} \todo{Graphs...}.

	\item \emph{Game theory models}: \label{bullet:GameTheory}In Game theory, competitors in a market are expected to be heavily influenced in their decisionmaking processes by the actions of other participants of the scenario. \todo{citation} In the case of our framework this manifests in the interaction between the different vendors in the market, where pricing decisions are often made interdependently. And while none of our rule based agents have been specifically designed to act according to game theoretic strategies, due to the fact that most of them consider thier competitors prices in their pricing decision, and due to the nature of Reinforcement Learning, behaviour according to Game theory can sometimes be observed. \todo{Need graphs to back this claim up! Especially price-graphs, moving up and down etc.} When it comes to training our Reinforcement Learning agents,

	\item \emph{Machine learning models}:

	\item \emph{Simulation models}:
\end{enumerate}


\section{Diverging from the realistic market}

We do not claim in any way that our simulation framework is exhaustive or complete. There are a number of parameters influencing market dynamics, small and great, which have not been modelled due to time constraints or their unpredictability in their affect on the market state or customer decisions. Seasonality of demand, customer retention and loyalty through branding and unforeseen effects on market dynamics such as global events (e.g. in recent times the Covid-19 pandemic) are just some of the places where our market simulation diverges from the real market. However, all of these factors are either very rare in nature (global, unforeseen events), only applicable to a subset of markets (brand loyalty) or can initially be discarded as their effect on market dynamics is predictable, making them a lower priority than other parameters (seasonality).
\todo{What effect does this have on our monitoring?}