% This file should contain the English abstract.
When scaling software projects, knowing when and how to start scaling maintainability is the key to success.\todo{Problem}
Maintainability manifests in many different ways, be it well documented API's, a comprehensive wiki or sensible tests.
Maintainability can be aided by automating many processes, as this takes strain away from developers and
lowers the barrier of entry for new team members looking to contribute.\todo{Background}

In our project, we employed many of these techniques, and this thesis aims to illustrate the goals and ideas behind
the processes involved.\todo{Objective} Primary focus is the journey from a single-file project to the pip-package `recommerce`
with its automated testing pipeline, code-style checks and comprehensive documentation using automated tools such as 
`Sphinx` and `interrogate`.\todo{Methodology}

Analysis of the process shows that while tests are useful in aiding developers to understand code it must always be
kept in mind that while tests make certain aspects of the software development process more accessible, they are
also to be maintained.\todo{Results - focuses on tests while leaving out documentation and pip (which comes in conclusion?!?)}

One of the biggest and most costly undertakings was the introduction of pip-packaging the project, which leads
to the conclusion that while building and maintaining a clear vision of the desired state of the product as early as possible
can lead to an overhead early in the process, it will also make the project grow more sustainably, leading to higher
maintainability in the long run.\todo{Conclusion}