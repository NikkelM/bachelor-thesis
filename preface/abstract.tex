Sustainable recommerce markets are growing faster than ever. In such markets, customers are incentiviced to re-sell their used products to businesses, which then refurbish and sell them on the secondary market. However, businesses now face the challenge of having to set three different prices for the same item: One price for the new item, one for its refurbished version and the price at which items are bought back from customers. Since these prices are heavily influenced by each other, traditional pricing methods become less effective. To solve this dynamic pricing problem, a simulation framework was built which can be used to train artificial vendors to set optimized prices using Reinforcement learning algorithms.
Before employing these trained agents on real markets, their fitness must be monitored and evaluated, as prices that are too high or too low can lead to high costs for the business. This thesis introduces a number of ways that such dynamic pricing agents can be monitored. We come to the conclusion that it is best to use a wide range of tools when evaluating different aspects of an agent's performance, from running large-scale simulations to monitoring small policy changes following shifting market states.