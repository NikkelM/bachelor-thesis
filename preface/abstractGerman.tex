Nachhaltige Recommerce-Märkte befinden sich in stetigem Wachstum. Dies stellt Unternehmen jedoch vor die neuartige Herausforderung, dasselbe Produkt mehrfach bepreisen zu müssen: Preise sowohl für die neue und generalüberholte Version sowie ein Ankaufpreis für gebrauchte Ware müssen gesetzt werden. Da diese Preise voneinander abhängig sind, greifen traditionelle Methoden der Preissetzung schlechter. Zur Lösung dieses dynamischen Bepreisungsproblems wurde eine Simulationsplattform gebaut, auf der mithilfe von Reinforcement-Learning Algorithmen maschinelle Verkäufer für den Einsatz in realen Märkten trainiert werden können. Bevor dies jedoch geschehen kann müssen die trainierten Modelle bezüglich ihrer Eignung überprüft und bewertet werden, da bereits der kleinste Fehler zu hohen Verlusten aufseiten des Unternehmens führen kann. Diese Arbeit führt Methoden und Tools ein, die für ein solches Monitoring verwendet werden können. Es wird festgestellt, dass die effektivsten Tools dabei... \todo{...todo}