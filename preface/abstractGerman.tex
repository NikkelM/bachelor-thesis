Nachhaltige Recommerce-Märkte befinden sich in stetigem Wachstum. Dies stellt Unternehmen jedoch vor die neuartige Herausforderung, dasselbe Produkt mehrfach bepreisen zu müssen: Preise sowohl für die neue und generalüberholte Version sowie ein Ankaufpreis für gebrauchte Ware müssen gesetzt werden. Da diese Preise voneinander abhängig sind, greifen traditionelle Methoden der Preissetzung schlechter. Zur Lösung dieses dynamischen Bepreisungsproblems wurde eine Simulationsplattform gebaut, auf der mithilfe von Reinforcement-Learning-Algorithmen maschinelle Verkäufer für den Einsatz in realen Märkten trainiert werden können. Bevor dies jedoch geschehen kann müssen die trainierten Modelle bezüglich ihrer Eignung überprüft und bewertet werden, da zu hoch oder niedrig angesetzte Preise zu hohen Verlusten aufseiten des Unternehmens führen kann. Diese Arbeit führt Tools ein, die für ein Monitoring solcher Modelle verwendet werden können. Wir stellen fest, dass die Nutzung möglichst diversifizierter Methoden, von der Simulation großangelegter Märkte bis zum Monitoring kleinster Verhaltensänderung aufgrund geänderter Marktzustände, die besten Ergebnisse bei der Bewertung einer Bepreisungsmethode liefert.