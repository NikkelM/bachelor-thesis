Nachhaltige Recommerce-Märkte befinden sich in stetigem Wachstum. Dies stellt Unternehmen jedoch vor die neuartige Herausforderung, dasselbe Produkt mehrfach bepreisen zu müssen: Preise sowohl für die neue und generalüberholte Version sowie ein Ankaufpreis gebrauchter Ware müssen gesetzt werden. Da diese Preise voneinander abhängig sind, greifen traditionelle Methoden der Preisfindung schlechter. Zur Lösung dieses dynamischen Bepreisungsproblems wurde eine Simulationsplattform gebaut, auf der mithilfe von Reinforcement-Learning Algorithmen künstliche Verkäufer \todo{Schönerer Name für "artificial vendors"?} für den Einsatz in realen Märkten trainiert werden können. Bevor dies jedoch geschehen kann müssen die trainierten Modelle auf Zuverlässigkeit und Robustheit \todo{Übersetze ich die beiden Begriffe, oder lasse ich sie englisch?} überprüft werden, da bereits der kleinste Fehler zu hohen Verlusten des Unternehmens führen kann. Diese Arbeit führt Methoden und Tools ein, die zu einem solchen Monitoring verwendet werden können. Es stellt sich heraus, dass die effektivsten Tools dabei... \todo{...todo}