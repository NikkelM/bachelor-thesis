% Here you introduce your topic to the reader.
\begin{jointwork}
	This thesis builds upon the bachelors project "Online Marketplace Simulation: A Testbed for Self-Learning Agents"
	of the Enterprise Platform and Integration Concepts research group at the Hasso-Plattner-Institute. Therefore, the project will
	be referenced and all examples and experiments, unless otherwise mentioned, will have been conducted using its framework.
\end{jointwork}

\section{Introduction to the Recommerce platform}

\subsection{The Circular Economy model}
The main goal of the aforementioned bachelors project was to develop a realistic online marketplace that simulates
a Circular Economy. A market is most commonly referred to as being a "Circular Economy" if it includes the three
activities of reduce, reuse and recycle \cite{circularEconomyDefinition}. This means that while in a classical
Linear Economy market each product is being sold once at its \emph{new price} and after use being thrown away,
in a Circular Economy, recycling and thereby waste reduction is a major focus. In our project, we modelled this by
adding two additional price channels, \emph{re-buy price} and \emph{used price}, to the pre-existing \emph{new price} of a product.

The \emph{re-buy price} is defined as the price a vendor is willing to pay a customer to buy back a used product, while the
\emph{used price} is defined as the price the vendor sets for products they previously bought back and now want to sell alongside new products
(whose price is defined by the \emph{new price}).

From now on, we will refer to the Circular Economy market simply as \emph{the market}.

\subsection{Using the simulated marketplace to train agents}

After the initial market was modelled the goal was to train agents using different reinforcement-learning algorithms
to dynamically set prices on this marketplace, both in monopolistic scenarios as well as in competition with rule-based vendors
which set prices following a strict set of pre-defined rules. \todo{Is the following sentence a footnote?}These rules can range from simply undercutting the
lowest competitor's price to more advanced techniques such as price-fixing and -gouging.\todo{find out if we have such competitors}

\begin{figure}
	\centering
	\includegraphics[height = 5 cm]{graphics/red_square_placeholder.png}\\[1 ex]
	\caption{The reinforcement-learning model in the context of our market.}
	\label{fig:IntroRLDiagram}
\end{figure}

Reinforcement-learning agents are trained through a process of trial-and-error. They interact with the market through an observable state
and an action which influences the following state. \Cref{fig:IntroRLDiagram} illustrates the RL-model in the context of our 
market. \todo{Create a nice diagram with the three prices}


\section{An Overview of Reinforcement-Learning}

Short description of Reinforcement learning:

- Agent interacts with the environment only on the basis of a state, which contains information about the
environment that the agent can use to decide on one of a number of specified actions it can take.

\section{Reliability and Robustness}


Agents to be trained for real-world use
Training in an isolated environment
Need to make sure they are "good"
What we want to offer with our framework
Determining the grade of an agent using monitoring