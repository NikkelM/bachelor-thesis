% This is where all the commands should go that you want to define yourself.
\newcommand*{\fullref}[1]{\hyperref[{#1}]{\ref*{#1} \nameref*{#1}}}

%% Code styling
\lstset{basicstyle=\linespread{1}\ttfamily\small,floatplacement=htbp,captionpos=t,abovecaptionskip=.5\baselineskip,belowcaptionskip=.5\baselineskip,upquote=true,showstringspaces=false,inputencoding=utf8,tabsize=4}
\lstset{literate={á}{{\'a}}1 {é}{{\'e}}1 {í}{{\'i}}1 {ó}{{\'o}}1 {ú}{{\'u}}1 {Á}{{\'A}}1 {É}{{\'E}}1 {Í}{{\'I}}1 {Ó}{{\'O}}1 {Ú}{{\'U}}1 {à}{{\`a}}1 {è}{{\`e}}1 {ì}{{\`i}}1 {ò}{{\`o}}1 {ù}{{\`u}}1 {À}{{\`A}}1 {È}{{\'E}}1 {Ì}{{\`I}}1 {Ò}{{\`O}}1 {Ù}{{\`U}}1 {ä}{{\"a}}1 {ë}{{\"e}}1 {ï}{{\"i}}1 {ö}{{\"o}}1 {ü}{{\"u}}1 {Ä}{{\"A}}1 {Ë}{{\"E}}1 {Ï}{{\"I}}1 {Ö}{{\"O}}1 {Ü}{{\"U}}1 {â}{{\^a}}1 {ê}{{\^e}}1 {î}{{\^i}}1 {ô}{{\^o}}1 {û}{{\^u}}1 {Â}{{\^A}}1 {Ê}{{\^E}}1 {Î}{{\^I}}1 {Ô}{{\^O}}1 {Û}{{\^U}}1 {œ}{{\oe}}1 {Œ}{{\OE}}1 {æ}{{\ae}}1 {Æ}{{\AE}}1 {ß}{{\ss}}1 {ű}{{\H{u}}}1 {Ű}{{\H{U}}}1 {ő}{{\H{o}}}1 {Ő}{{\H{O}}}1 {ç}{{\c c}}1 {Ç}{{\c C}}1 {ø}{{\o}}1 {å}{{\r a}}1 {Å}{{\r A}}1 {€}{{\EUR}}1 {£}{{\pounds}}1 {~}{{\textasciitilde}}1 {-}{{-}}1 }
\makeatletter\renewcommand\verbatim@font{\ttfamily}\makeatother
\makeatletter\renewcommand\lstinline[1][]{ \errmessage{In diesem Template bitte die 'code'-Umgebung nutzen (an Stelle von 'lstinline').} }\makeatother
% \code-Umgebung mit Silbentrennung (Alternative für lstinline)
\newcommand{\code}[1]{\texttt{\selectlanguage{english}#1}}

% normalfont comment boxes (for listings)
\lstset{escapeinside={(*@}{@*)}}
\newcommand{\commentbox}[2][,] { %
	\begin{tikzpicture}[overlay,auto,>=latex] %
	\normalfont %
	\node[anchor=south] (target) {}; %
	\node[right=of target,align=left,anchor=west,#1] (box) { #2 }; %
	\draw[thin] (box.south west) |- (box.north west); %
	\draw[->,thin] (box.south west) |- (target.east); %
	\end{tikzpicture} %
}

\colorlet{punct}{red!60!black}
\definecolor{background}{HTML}{EEEEEE}
\definecolor{delim}{RGB}{20,105,176}
\colorlet{numb}{magenta!60!black}